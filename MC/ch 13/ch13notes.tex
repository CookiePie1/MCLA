\documentclass{article}
\title{Chapter 13 Notes - MC} % title - chapter number
\author{John Yang}
\usepackage{amsmath}
\usepackage{amssymb}
\usepackage[margin=1in, letterpaper]{geometry}
\usepackage{outlines}
\setcounter{section}{+12} % chapter number minus 1
\usepackage{mathtools}
\DeclarePairedDelimiter\set\{\}
\usepackage{hyperref}
\hypersetup{
	colorlinks=true,
	linkcolor=blue,
	filecolor=magenta,      
	urlcolor=cyan,
}
\usepackage{tocloft}
\renewcommand\cftsecfont{\normalfont}
\renewcommand\cftsecpagefont{\normalfont}
\renewcommand{\cftsecleader}{\cftdotfill{\cftsecdotsep}}
\renewcommand\cftsecdotsep{\cftdot}
\renewcommand\cftsubsecdotsep{\cftdot}

\begin{document}
    \maketitle
    \tableofcontents
    \section{Vectors and the Geometry of Space} % chapter title
    \subsection{Three-Dimensional Coordinate Systems} % section topic
    \begin{outline}
        \1 Coordinates - \((x,y,z)\)
        \1 3D space is split into octants
        \1 Projections - easiest way to visualize is that the object/shape/line/point is in a glass box. If you look at the box from the chosen plane or angle and trace a 2D outline of it, it is the projection. 
        \1 The Cartesian product \(\mathbb{R}\times\mathbb{R}\times\mathbb{R}=\{(x,y,z)|x,y,z\in\mathbb{R}\}\) is the set of all ordered triples of real numbers, denoted by \(\mathbb{R}^3\)
        \1 Distance formula in three dimensions: \[|P_1P_2|=\sqrt{(x_2-x_1)^2+(y_2-y_1)^2+(z_2-Z_1)^2}\]
        \1 Sphere - set of all points in 3D space a certain distance from the center. 
            \2 Sphere with center \(C(h,k,l)\) and radius $r$ is given by: \[(x-h)^2+(y-k)^2+(z-l)^2=r^2\]
            \2 Sphere at the origin is given by: \[x^2+y^2+z^2=r^2\]
        
    \end{outline}
    \subsection{Vectors}
    \begin{outline}
        \1 Vectors - values with magnitudes and directions
        \1 vectors are expressed in boldface and/or with an arrow over the letter: \(\mathbf{a}\), \(\vec{a}\), \(\vec{\mathbf{a}}\)
        \1 the magnitude of a vector is expressed: \(|\mathbf{a}|\)
        \1 Vector addition - head to tail, take the magnitude of the resultant vector from the beginning of the first vector to the end of the second vector 
        \1 Scalar multiplication - multiply a vector by a scalar: the magnitude is changed by a factor of $c$ but the direction stays the same. 

    \end{outline}
    \subsection{The Dot Product}
    \begin{outline}
        \1 
    \end{outline} 
    \subsection{The Cross Product}
    \begin{outline}
        \1 
    \end{outline}
    \subsection{Equations of Lines and Planes}
    \begin{outline}
        \1 
    \end{outline}
    \subsection{Cylinders and Quadric Surfaces}
    \begin{outline}
        \1 
    \end{outline}
\end{document}