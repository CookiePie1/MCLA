\documentclass{article}
\title{Chapter 13 Notes - MC} % title - chapter number
\author{John Yang}
\usepackage{amsmath}
\usepackage{amssymb}
\usepackage[margin=1in, letterpaper]{geometry}
\usepackage{outlines}
\setcounter{section}{+12} % chapter number minus 1
\usepackage{mathtools}
\DeclarePairedDelimiter\set\{\}
\usepackage{hyperref}
\hypersetup{
	colorlinks=true,
	linkcolor=blue,
	filecolor=magenta,      
	urlcolor=cyan,
}
\usepackage{tocloft}
\renewcommand\cftsecfont{\normalfont}
\renewcommand\cftsecpagefont{\normalfont}
\renewcommand{\cftsecleader}{\cftdotfill{\cftsecdotsep}}
\renewcommand\cftsecdotsep{\cftdot}
\renewcommand\cftsubsecdotsep{\cftdot}

\begin{document}
    \maketitle
    \tableofcontents
    \section{Vectors and the Geometry of Space} % chapter title
    \subsection{Three-Dimensional Coordinate Systems} % section topic
    \begin{outline}
        \1 Coordinates - \((x,y,z)\)
        \1 3D space is split into octants
        \1 Projections - easiest way to visualize is that the object/shape/line/point is in a glass box. If you look at the box from the chosen plane or angle and trace a 2D outline of it, it is the projection. 
        \1 The Cartesian product \(\mathbb{R}\times\mathbb{R}\times\mathbb{R}=\{(x,y,z)|x,y,z\in\mathbb{R}\}\) is the set of all ordered triples of real numbers, denoted by \(\mathbb{R}^3\)
        \1 Distance formula in three dimensions: \[|P_1P_2|=\sqrt{(x_2-x_1)^2+(y_2-y_1)^2+(z_2-Z_1)^2}\]
        \1 Sphere - set of all points in 3D space a certain distance from the center. 
            \2 Sphere with center \(C(h,k,l)\) and radius $r$ is given by: \[(x-h)^2+(y-k)^2+(z-l)^2=r^2\]
            \2 Sphere at the origin is given by: \[x^2+y^2+z^2=r^2\]
        
    \end{outline}
    \subsection{Vectors}
    \begin{outline}
        \1 Vectors - values with magnitudes and directions
        \1 vectors are expressed in boldface and/or with an arrow over the letter: \(\mathbf{a}\), \(\vec{a}\), \(\vec{\mathbf{a}}\)
        \1 the magnitude of a vector is expressed: \(|\mathbf{a}|\)
        \1 Vector addition - head to tail, take the magnitude of the resultant vector from the beginning of the first vector to the end of the second vector 
        \1 Definition of scalar multiplication: If $c$ is a scalar and $\mathbf{v}$ is a vector, then the scalar multiple \(c\mathbf v\) is the vector whose lengthe is \(|c|\) times the length of \(\mathbf v\) and whose direction is the same as \(\mathbf v\) if \(c>0\) and is opposite to \(\mathbf v\) if \(c<0\). If \(c=0\) or \(\mathbf{v=0}\), then \(c\mathbf{v=0}\). 
        \1 Components of a vector: Equation 1: Given the points \(A(x_1,y_1,z_1)\) and \(B(x_2,y_2,z_2)\), the vector \(\mathbf a\) with representation \(\vec{AB}\) is: \[\mathbf a=\langle x_2-x_1, y_2-y_1,z_2-z_1\rangle\]
        \1 The length of the two-dimensional vector \(\mathbf a=\langle a_1,a_2\rangle\) is \[|\mathbf a|=\sqrt{a^2_1+a_2^2}\]
        \1 The length of the three-dimensional vector \(\mathbf a=\langle a_1,a_2,a_3\rangle\) is \[|\mathbf a|=\sqrt{a^2_1+a_2^2+a_3^2}\]
        \1 If \(\mathbf a=\langle a_1,a_2\rangle\) and \(\mathbf b=\langle b_1,b_2\rangle\), then \[\mathbf a+\mathbf b=\langle a_1+b_1,a_2+b_2\rangle\qquad\qquad\mathbf a-\mathbf b=\langle a_1-b_1,a_2-b_2\rangle\]\[c\mathbf a=\langle ca_1,ca_2\rangle \]
        \1 For three dimensional vectors, \[\langle a_1,a_2,a_3\rangle+\langle b_1, b_2,b_3\rangle=\langle a_1+b_1, a_2+b_2,a_3+b_3\rangle\]\[\langle a_1,a_2,a_3\rangle-\langle b_1, b_2,b_3\rangle=\langle a_1-b_1, a_2-b_2, a_3-b_3\rangle\]\[c\langle a_1,a_2,a_3\rangle=\langle ca_1,ca_2,ca_3\rangle\]
        \1 \(V_2\) is the set of all 2-D vectors. \(V_3\) is the set of all 3-D vectors. \(V_n\) is the set of all $n$-dimensional vectors. 
        \1 Properties of vectors. If \(\mathbf a\),\(\mathbf{b}\), and\(\mathbf c\) are vectors in \(V_n\) and $c$ and $d$ are scalars, then: 
            \2 \(\mathbf a+\mathbf b=\mathbf b+\mathbf a\)
            \2 \(\mathbf a+(\mathbf b+\mathbf c)=(\mathbf a+\mathbf b)+\mathbf c\)
            \2 \(\mathbf a+\mathbf 0=\mathbf a\)
            \2 \(\mathbf a+(-\mathbf a)=0\)
            \2 \(c(\mathbf a+\mathbf b)=c\mathbf a+c\mathbf b\)
            \2 \((c+d)\mathbf a=c\mathbf a+d\mathbf a\)
            \2 \((cd)\mathbf a=c(d\mathbf a)\)
            \2 \(1\mathbf a=\mathbf a\)
        \1 Unit vectors: \(\mathbf i=\langle 1,0,0\rangle\), \(\mathbf j=\langle 0,1,0\rangle\), \(\mathbf k=\langle 0,0,1\rangle\)
        \1 Use unit vectors to express components: \(\mathbf a=\langle a_1,a_2,a_3\rangle=a_1\mathbf i+a_2\mathbf j+a_3\mathbf k\)
        \1 General unit vector expresses direction \[\mathbf u=\dfrac{\mathbf a}{|\mathbf a|}\]
        \1 

    \end{outline}
    \subsection{The Dot Product}
    \begin{outline}
        \1 Definition 1: If \(\mathbf a=\langle a_1,a_2,a_3\rangle\), and \(\mathbf b=\langle b_1,b_2,b_3\rangle\), then the dot product of \(\mathbf a\) and \(\mathbf b\) is the scalar \(\mathbf{a\cdot b}\) given by \[\mathbf{a\cdot b}=a_1b_1+a_2b_2+a_3b_3\]
        \1 Other names for dot product: scalar product, inner product
        \1 Properties of the dot product: If \(\mathbf a\), \(\mathbf b\), \(\mathbf c\) are vectors in \(V_3\) and $c$ is a scalar, then: 
            \2 \(\mathbf{a\cdot a}=|\mathbf a|^2\)
            \2 \(\mathbf{a\cdot b}=\mathbf{b\cdot a}\)
            \2 \(\mathbf{a\cdot}(\mathbf{b+c})=\mathbf{a\cdot b+a\cdot c}\)
            \2 \((c\mathbf a)\mathbf{\cdot b}=c(\mathbf{a\cdot b})=\mathbf{a\cdot}(c\mathbf b)\)
            \2 \(\mathbf{0\cdot a}=0\)
        \1 Theorem 3: If \(\theta\) is the angle between the vectors \(\mathbf a\) and \(\mathbf b\), then \[\mathbf{a\cdot b}=|\mathbf a||\mathbf b|\cos\theta\]
        \1 Corollary 6: If \(\theta\) is the angle between the nonzero vectors \(\mathbf a\) and \(\mathbf b\), then \[\cos\theta=\dfrac{\mathbf{a\cdot b}}{|\mathbf a||\mathbf b|}\]
        \1 Equation 7: Two vectors \(\mathbf a\) and \(\mathbf b\),are orthogonal IFF \(\mathbf{a\cdot b}=0\)
        \1 Direction angles of a nonzero vector \(\mathbf a\) are the angles \(\alpha\), \(\beta\), and \(\gamma\) that \(\mathbf a\) makes with the positive \(x\)-, \(y\)-, and \(z\)-axes respectively. The cosines of the direction angles are called direction cosines. 
        \1 Direction angles are given by: \[\cos\alpha=\dfrac{a_1}{|\mathbf a|}\qquad\cos\beta=\dfrac{a_2}{|\mathbf a|}\qquad\cos\gamma=\dfrac{a_3}{|\mathbf a|}\] and, \[\dfrac{1}{|\mathbf a|}\mathbf a=\langle\cos\alpha,\cos\beta,\cos\gamma\rangle\]
        \1 Projections: think of it like a shadow. 
        \1 Scalar projection of \(\mathbf b\) onto \(\mathbf a\) (aka component of \(\mathbf b\) along \(\mathbf{a}\)): \[\text{comp}_{\mathbf a}\mathbf b=\dfrac{\mathbf{a\cdot b}}{|\mathbf a|}\]
        \1 Vector projection of \(\mathbf b\) onto \(\mathbf a\): \[\text{proj}_{\mathbf a}\mathbf b=\left(\dfrac{\mathbf{a\cdot b}}{|\mathbf a|}\right)\dfrac{\mathbf a}{|\mathbf a|}=\dfrac{\mathbf{a\cdot b}}{|\mathbf a|^2}\mathbf a\]

    \end{outline} 
    \subsection{The Cross Product}
    \begin{outline}
        \1 Definition of the cross product: If \(\mathbf a=\langle a_1,a_2,a_3\rangle\), and \(\mathbf b=\langle b_1,b_2,b_3\rangle\), then the cross product of \(\mathbf a\) and \(\mathbf b\) is the vector \[\mathbf{a\times b}=\langle a_2b_3-a_3b_2,a_3b_1-a_1b_3,a_1b_2-a_2b_1\rangle\]
        \1 Cross product is also called vector product or external product. 
        \1 Cross product is only defined when both \(\mathbf a\) and \(\mathbf b\) are 3-D vectors. 
        \1 Determinant form of the cross product: \[\mathbf{a\times b}=\begin{vmatrix}
            \mathbf i & \mathbf j & \mathbf k\\
            a_1 & a_2 & a_3\\
            b_1 & b_2 & b_3
        \end{vmatrix}\]
        \1 Theorem 8: The vector \(\mathbf{a\times b}\) is orthogonal to both \(\mathbf a\) and \(\mathbf b\). 
        \1 Direction of the external product: use curling rhr - fingers curl from \(\mathbf a\) to \(\mathbf b\), thumb is the direction of the cross product. 
        \1 Theorem 9: magnitude of the cross product: If \(\theta\) is the angle between \(\mathbf a\) and \(\mathbf b\) (so \(0\leq\theta\leq\pi\)), then \[|\mathbf{a\times b}|=|\mathbf a||\mathbf b|\sin\theta\]
        \1 Corollary 10: Two nonzero vector \(\mathbf a\) and \(\mathbf b\) are parallel IFF \[\mathbf{a\times b}=\mathbf 0\]
        \1 The length of the cross product \(\mathbf{a\times b}\) is equal to the area of the parallelogram determined by \(\mathbf a\) and \(\mathbf b\). 
        \1 Cross products of unit vectors: \[\mathbf{i\times j=k} \qquad \qquad \mathbf{j\times k=i} \qquad \qquad \mathbf{k\times i=j}\]\[\mathbf{j\times i=-k} \qquad \qquad \mathbf{k\times j=-i} \qquad \qquad \mathbf{i\times k=-j}\]
        \1 Cross product is neither commutative nor associative. 
        \1 Properties of the cross product: If \(\mathbf a\), \(\mathbf b\), and \(\mathbf c\) are vectors and \(c\) is a scalar, then: 
            \2 \(\mathbf{a\times b=-b\times a}\)
            \2 \((c\mathbf a)\times \mathbf b=c(\mathbf{a\times b})=\mathbf a\times(c\mathbf b)\)
            \2 \(\mathbf{a}\times(\mathbf{b+c})=\mathbf{a\times b+a\times c}\)
            \2 \((\mathbf{a+b})\times \mathbf c=\mathbf{a\times c+b\times c}\)
            \2 \(\mathbf{a\cdot(b\times c)}=\mathbf{(a\times b)\cdot c}\)
            \2 \(\mathbf{a\times(b\times c)}=\mathbf{(a\cdot c)b-(a\cdot b)c}\)
        \1 Scalar triple product of vectors \(\mathbf a\), \(\mathbf b\), and \(\mathbf c\): \[\mathbf{a\cdot(b\times c)}=\begin{vmatrix}
            a_1 & a_2 & a_3\\
            b_1 & b_2 & b_3 \\
            c_1 & c_2 & c_3
        \end{vmatrix}\]
        \1 The scalar triple product is the volume of the parallelepiped determined by vectors \(\mathbf a\), \(\mathbf b\), and \(\mathbf c\) and is given by: \[V=|\mathbf{a\cdot(b\times c)}|\]

    \end{outline}
    \subsection{Equations of Lines and Planes}
    \begin{outline}
        \1 Let the line $L$ be any line in 3D space, which is determined when there is a point on $L$, \(P_0(x_0,y_0,z_0)\), and we know the direction of $L$. Let \(P(x,y,z)\) be any point on $L$ and let \(\mathbf r_0\) and \(\mathbf r\) be the position vectors of \(P_0\) and \(P\). Let \(\mathbf v\) be a vector parallel to $L$. The vector equation of $L$ is given by \[\mathbf r=\mathbf r_0+t\mathbf v\] where $t$ is a parameter. 
        \1 Parametric equations for a line $L$ through the point \((x_0,y_0,z_0)\) and parallel to the direction vector \(\langle a,b,c\rangle\) are: \[x=x_0+at\qquad y=y_0+bt\qquad z=z_0+ct\]
        \1 Symmetric equations of $L$: \[\dfrac{x-x_0}{a}=\dfrac{y-y_0}{b}=\dfrac{z-z_0}{c}\]
        \1 The line segment from \(\mathbf{r_0}\) to \(\mathbf{r_1}\) is given by the vector equation \[\mathbf r(t)=(1-t)\mathbf{r_0}+t\mathbf{r_1}\quad 0\leq t\leq 1\]
        \1 A plane is determined by a point \(P_0(x_0,y_0,z_0)\) in the plane and an orthogonal normal vector \(\mathbf n\). The vector equation of the plane is given by \[\mathbf{n\cdot}(\mathbf{r}-\mathbf{r_0})=0\] or \[\mathbf{n\cdot r}=\mathbf{n\cdot r_0}\]
        \1 Scalar equation of the plane through point \(P_0(x_0,y_0,z_0)\) with normal vector \(\mathbf n=\langle a,b,c\rangle\) is \[a(x-x_0)+b(y-y_0)+c(z-z_0)=0\]
        \1 The distance $D$ from any point \(P_1(x_1,y_1,z_1)\) to a plane \(ax+by+cz+d=0\) is given by \[D=\dfrac{|ax_1+by_1+cz_1+d|}{\sqrt{a^2+b^2+c^2}}\]

    \end{outline}
    \subsection{Cylinders and Quadric Surfaces}
    \begin{outline}
        \1 A cylinder is a surface that consists of all lines (called rulings) that are parallel to a given line and pass through a given plane curve. 
        \1 A quadric surface is the graph of a second-degree equation in three variables \(x\), \(y\), and \(z\). The most general form of a quadric surface is: \[Ax^2+By^2+Cz^2+Dxy+Eyz+Fxz+Gx+Hy+Iz+J=0\]
        \1 It can also take one of two standard forms: \[Ax^2+By^2+Cz^2+J=0\qquad\qquad\text{or}\qquad\qquad Ax^2+By^2+Iz=0\]
    \0 Common quadric surfaces (see page 877 for images): 
        \1 Ellipsoid - all traces are ellipses. Equation is given by: \[\dfrac{x^2}{a^2}+\dfrac{y^2}{b^2}+\dfrac{z^2}{c^2}=1\] If \(a=b=c\), then the ellipsoid is a sphere. 
        \1 Elliptic paraboloid: horizontal traces are ellipses and vertical traces are parabolas. The variable raised to the first power indicates the axis of the paraboloid. \[\dfrac{z}{c}=\dfrac{x^2}{a^2}+\dfrac{y^2}{b^2}\]
        \1 Hyperbolic paraboloid: Horizontal traces are hyperbolas and vertical traces are parabolas. \[\dfrac{z}{c}=\dfrac{x^2}{a^2}-\dfrac{y^2}{b^2}\]
        \1 Cone: Horizontal traces are ellipses. Vertical traces in the planes \(x=k\) and \(y=k\) are hyperbolas if \(k\neq 0\) but are pairs of lines if \(k=0\). \[\dfrac{z^2}{c^2}=\dfrac{x^2}{a^2}+\dfrac{y^2}{b^2}\]
        \1 Hyperboloid of one sheet: Horizontal traces are ellipses and vertical traces are hyperbolas. The axis of symmetry corresponds to the variable whose coefficient is negative. \[\dfrac{x^2}{a^2}+\dfrac{y^2}{b^2}-\dfrac{z^2}{c^2}=1\]
        \1 Hyperboloid of two sheets: Horizontal traces in \(z=k\) are ellipses if \(k>c\) or \(k<-c\). Vertical traces are hyperbolas. The two minus signs indicate the two sheets. \[-\dfrac{x^2}{a^2}-\dfrac{y^2}{b^2}+\dfrac{z^2}{c^2}=1\]
        
    \end{outline}
\end{document}