\documentclass{article}
\title{Chapter 14 Notes - MC} % title - chapter number
\author{John Yang}
\usepackage{amsmath}
\usepackage[margin=1in, letterpaper]{geometry}
\usepackage{outlines}
\setcounter{section}{+13} % chapter number minus 1
\usepackage{mathtools}
\DeclarePairedDelimiter\set\{\}
\usepackage{hyperref}
\hypersetup{
	colorlinks=true,
	linkcolor=blue,
	filecolor=magenta,      
	urlcolor=cyan,
}
\usepackage{tocloft}
\renewcommand\cftsecfont{\normalfont}
\renewcommand\cftsecpagefont{\normalfont}
\renewcommand{\cftsecleader}{\cftdotfill{\cftsecdotsep}}
\renewcommand\cftsecdotsep{\cftdot}
\renewcommand\cftsubsecdotsep{\cftdot}
 
\begin{document}
    \maketitle
    \tableofcontents
    \section{Vector Functions} % chapter title
    \subsection{Vector Functions and Space Curves} % section topic
    \begin{outline}
        \1 vector-valued functions/vector functions - a function whose domain is a set of real numbers and whose range is a set of vectors. Written in terms of its components as a parametric equation: \[\mathbf r(t)=\langle f(t),g(t),h(t)\rangle=f(t)\mathbf i+g(t)\mathbf j+h(t)\mathbf k\]
        \1 Limits of a vector function: If \(\mathbf r(t)=\langle f(t),g(t),h(t)\rangle\), then \[\lim_{t\to a}\mathbf r(t)=\left\langle\lim_{t\to a}f(t),\lim_{t\to a}g(t),\lim_{t\to a}h(t)\right\rangle\] provided the limits of the component functions exist. 
        \1 Space curves: suppose that $f$, $g$, and $h$ are continuous real-valued functions on an interval $I$. Then the space curve is the set $C$ of all points \((x,y,z)\) in space, where \[x=f(t)\qquad y=g(t)\qquad z=h(t)\] and $t$ varies throughout the interval $I$. 
    \end{outline}
    \subsection{Derivatives and Integrals of Vector Functions}
    \begin{outline}
        \1 derivative of a vector-valued function: \[\dfrac{d\mathbf r}{dt}=\mathbf r'(t)=\lim_{h\to 0}\dfrac{\mathbf r(t+h)-\mathbf r(t)}{h}\] if the limit exists. 
        \1 Theorem 2: If \(\mathbf r(t)=\langle f(t),g(t),h(t)\rangle=f(t)\mathbf i+g(t)\mathbf j+h(t)\mathbf k\), where $f$, $g$, and $h$ are differentiable functions, then \[\mathbf r'(t)=\langle f'(t),g'(t),h'(t)\rangle =f'(t)\mathbf i+g'(t)\mathbf j+h'(t)\mathbf k\]
        \1 Theorem 3: Suppose $\mathbf u$ and $\mathbf v$ are differentiable vector functions, $c$ is a scalar, and $f$ is a real valued function. Then: 
            \2 \[\dfrac{d}{dt}[\mathbf u(t)+\mathbf v(t)]=\mathbf u'(t)+\mathbf v'(t)\]
            \2 \[\dfrac{d}{dt}[c\mathbf u(t)]=c\mathbf u'(t)\]
            \2 \[\dfrac{d}{dt}[f(t)\mathbf u(t)]=f'(t)\mathbf u(t)+f(t)\mathbf u'(t)\]
            \2 \[\dfrac{d}{dt}[\mathbf u(t)\cdot\mathbf v(t)]=\mathbf u'(t)\cdot\mathbf v(t)+\mathbf u(t)\cdot\mathbf v'(t)\]
            \2 \[\dfrac{d}{dt}[\mathbf u(t)\times\mathbf v(t)]=\mathbf u'(t)\times\mathbf v(t)+\mathbf u(t)\times\mathbf v'(t)\]
            \2 \[\dfrac{d}{dt}[\mathbf u(f(t))]=f'(t)\mathbf u'(f(t))\] (chain rule)
        \1 Integral of a vector function: \[\int^b_a\mathbf r(t)dt=\left(\int^b_af(t)dt\right)\mathbf i+\left(\int^b_ag(t)dt\right)\mathbf j\left(\int^b_ah(t)dt\right)\mathbf k\]
        
    \end{outline}
    \subsection{Arc Length and Curvature}
    \begin{outline}
        \1 Length of a curve in 3D space: \[L=\int^b_a\sqrt{[f'(t)]^2+[g'(t)]^2+[h'(t)]^2}dt=\int^b_a\sqrt{\left(\dfrac{dx}{dt}\right)^2+\left(\dfrac{dy}{dt}\right)^2+\left(\dfrac{dx}{dt}\right)^2}dt=\int^b_a|\mathbf r'(t)|dt\]
        \1 curvature of a curve is defined as: \[\kappa=\left|\dfrac{d\mathbf T}{ds}\right|\] where $\mathbf T$ is the unit tangent vector. 
        \1 \[\kappa(t)=\dfrac{|\mathbf T'(t)|}{|\mathbf r'(t)|}\]
        \1 Theorem 10: The curvature of the curve given by the vector function $\mathbf r$ is \[\kappa(t)=\dfrac{|\mathbf r'(t)\times\mathbf r''(t)|}{|\mathbf r'(t)|^3}\]
        \1 Equations for unit tangent, unit normal and binormal vectors, and curvature: \[\mathbf T(t)=\dfrac{\mathbf r'(t)}{|\mathbf r'(t)|}\qquad \mathbf N(t)=\dfrac{\mathbf T'(t)}{|\mathbf T'(t)|}\qquad \mathbf B(t)=\mathbf T(t)\times\mathbf N(t)\]\[\kappa=\left|\dfrac{d\mathbf T}{ds}\right|=\dfrac{|\mathbf T'(t)|}{|\mathbf r'(t)|}=\dfrac{|\mathbf r'(t)\times r''(t)|}{|\mathbf r'(t)|^3}\]

    \end{outline}
    \subsection{Motion in Space: Velocity and Acceleration}
    \begin{outline}
        \1 Velocity: \[\mathbf v(t)=\mathbf r'(t)\]
        \1 speed is the magnitude of velocity. 
        \1 Parametric equations of trajectory: \[x=(v_0\cos\alpha)t\qquad y=(v_0\sin\alpha)t-\frac{1}{2}gt^2\]
        \1 Tangential and normal components of acceleration: \[\mathbf a=v'(\mathbf T)+\kappa v^2\mathbf N\]
        \1 Kepler's laws: 
            \2 A planet revolves around the sun in an elliptical orbit with the sun at one focus. 
            \2 The line joining the sun to a planet sweeps out equal areas in equal times. 
            \2 The square of the period of revolution of a planet is proportional to the cube of the length of the major axis of orbit. 
    \end{outline}
\end{document}