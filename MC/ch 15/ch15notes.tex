\documentclass{article}
\title{Chapter 15 Notes - MC} % title - chapter number
\author{John Yang} 
\usepackage{amsmath}
\usepackage{amssymb}
\usepackage[margin=1in, letterpaper]{geometry}
\usepackage{outlines}
\setcounter{section}{+14} % chapter number minus 1
\usepackage{mathtools}
\usepackage{physics}
\DeclarePairedDelimiter\set\{\}
\usepackage{hyperref}
\hypersetup{
	colorlinks=true,
	linkcolor=blue,
	filecolor=magenta,      
	urlcolor=cyan,
}
\usepackage{tocloft}
\renewcommand\cftsecfont{\normalfont}
\renewcommand\cftsecpagefont{\normalfont}
\renewcommand{\cftsecleader}{\cftdotfill{\cftsecdotsep}}
\renewcommand\cftsecdotsep{\cftdot}
\renewcommand\cftsubsecdotsep{\cftdot}

\begin{document}
    \maketitle
    \tableofcontents
    \section{Partial Derivatives} % chapter title
    \subsection{Functions of Several Variables} % section topic
    \begin{outline}
        \1 A function $f$ of two variables is a rule that assigns to each ordered pair of real numbers \((x,y)\) in a set $D$ a unique real number denoted by \(f(x,y)\). The set $D$ is the domain of $f$ and its range is the set of values that $f$ takes on, that is \({f(x,y)|(x,y)\in D}\)
        \1 If $f$ is a function of two variables with domain $D$, then the graph of $f$ is the set of all points \((x,y,z)\) in \(\mathbb{R}^3\) such that \(z=f(x,y)\) and \((x,y)\) is in $D$. 
        \1 The level curves of a function $f$ of two variables are the curves with equations \(f(x,y)=k\), where $k$ is a constant (in the range of $f$). 
            \2 A level curve is the set ofa ll points in the domain of $f$ at which $f$ takes on a given value $k$. (Think of countour maps, equipotential lines)


    \end{outline}
    \subsection{Limits and Continuity}
    \begin{outline}
        \1 Let $f$ be a function of two variables whose domain $D$ includes points arbitrarily close to \((a,b)\). Then we say that the limit of \(f(x,y)\) as \(x,y\) approaches \((a,b)\) is $L$ and we write \[\lim_{(x,y)\to(a,b)}f(x,y)=L\] if for every number \(\varepsilon>0\) there is a corresponding number \(\delta>0\) such that \[\text{if}\qquad(x,y)\in D\qquad\text{and}\qquad 0<\sqrt{(x-a)^2+(y-b)^2}<\delta\qquad\text{then}\qquad|f(x,y)-L|<\varepsilon\]
        \1 If \(f(x,y)\to L_1\) as \((x,y)\to(a,b)\) along a path $C_1$ and \(f(x,y)\to L_2\) as \((x,y)\to(a,b)\) along a path $C_2$, where \(L_1\neq L_2\), then \(\lim_{(x,y)\to(a,b)}f(x,y)\) does not exist. 
        \1 A function $f$ of two variables is called continuous at \((a,b)\) if \[\lim_{(x,y)\to(a,b)f(x,y)=f(a,b)}\] We say $f$ is continuous on $D$ if $f$ is continuous at every point \((a,b)\) in $D$. 
        \1 If $f$ is defined on a subset $D$ of \(\mathbb{R}^n\), then \(\lim_{\mathbf x\to\mathbf a}f(\mathbf x)=L\) means that for every number \(\varepsilon>0\) there is a corresponding number \(\delta>0\) such that \[\text{if}\qquad x\in D\qquad\text{and}\qquad 0<|\mathbf x-\mathbf a|<\delta \qquad\text{then}\qquad |f(\mathbf x)-L|<\varepsilon\]
    \end{outline}
    \subsection{Partial Derivatives}
    \begin{outline}
        \1 If $f$ is a function of two variables, its partial derivatives are the functions $f_x$ and $f_y$ defined by \[f_x(x,y)=\lim_{h\to0}\dfrac{f(x+h,y)-f(x,y)}{h}\]\[f_y(x,y)=\lim_{h\to0}\dfrac{f(x,y+h)-f(x,y)}{h}\]
        \1 If \(z=f(x,y)\), we write \[f_x(x,y)=f_x=\dfrac{\partial f}{\partial x}=\dfrac{\partial}{\partial x}f(x,y)=\dfrac{\partial z}{\partial x}=f_1=D_1f=D_xf\]\[f_y(x,y)=f_y=\dfrac{\partial f}{\partial y}=\dfrac{\partial}{\partial y}f(x,y)=\dfrac{\partial z}{\partial y}=f_2=D_2f=D_yf\]
        \1 Finding partial derivatives of \(z=f(x,y)\):
            \2 To find $f_x$, regard $y$ as a constant and differentiate \(f(x,y)\) with respect to $x$. 
            \2 To find $f_y$, regard $x$ as a constant and differentiate \(f(x,y)\) with respect to $y$. 
        \1 Clairaut's Theorem: suppose $f$ is defined on a disk $D$ that contains the point \((a,b)\). If the functions \(f_{xy}\) and \(f_{yx}\) are both continuous on $D$, then \[f_{xy}(a,b)=f_{yx}(a,b)\]
        \1 3D Laplace Equation: \[\dfrac{\partial^2u}{\partial x^2}+\dfrac{\partial^2u}{\partial y^2}+\dfrac{\partial^2u}{\partial z^2}=0\]

    \end{outline}
    \subsection{Tangent Planes and Linear Approximations}
    \begin{outline}
        \1 Suppose $f$ has continuous partial derivatives. An equation of the tangent plane to the surface \(z=f(x,y)\) at the point \(P(x_0,y_0,z_0)\) is \[z-z_0=f_x(x_0,y_0)(x-x_0)+f_y(x_0,y_0)(y-y_0)\]
        \1 If \(z=f(x,y)\), then $f$ is differentiable at \((a,b)\) if \(\Delta z\) can be expressed in the form \[\Delta z=f_x(a,b)\Delta x+f_y(a,b)\Delta y+\varepsilon_1\Delta x+\varepsilon_2\Delta y\] where \(\varepsilon_1\) and \(\varepsilon_2\to 0\) as \((\Delta x,\Delta y)\to(0,0)\). 
        \1 Theorem: If the partial derivatives \(f_x\) and \(f_y\) exist near \((a,b)\) and are continuous at \((a,b)\), then $f$ is differentiable at \((a,b)\). 
        \1 The total differential $dz$ is defined by: \[dz=f_x(x,y)dx+f_y(x,y)dy=\dfrac{\partial z}{\partial x}dx+\dfrac{\partial z}{\partial y}dy\]

    \end{outline}
    \subsection{The Chain Rule}
    \begin{outline}
        \1 Case 1: suppose that \(z=f(x,y)\) is a differentiable function of $x$ and $y$, where \(x=g(t)\) and \(y=h(t)\) are both differentiable functions of $t$. Then $z$ is a differentiable function of $t$ and \[\dfrac{dz}{dt}=\dfrac{\partial f}{\partial x}\dfrac{dx}{dt}+\dfrac{\partial f}{\partial y}\dfrac{dy}{dt}\] or \[\dfrac{dz}{dt}=\dfrac{\partial z}{\partial x}\dfrac{dx}{dt}+\dfrac{\partial z}{\partial y}\dfrac{dy}{dt}\]
        \1 Case 2: Suppose that \(z=f(x,y)\) is a differentiable function of $x$ and $y$, where \(x=g(s,t)\) and \(y=h(s,t)\) are differentiable functions of $s$ and $t$. Then \[\pdv{z}{s}=\pdv{z}{x}\pdv{x}{s}+\pdv{z}{y}\pdv{y}{s}\qquad\pdv{z}{t}=\pdv{z}{x}\pdv{x}{t}+\pdv{z}{y}\pdv{y}{t}\]
        \1 General Version: Suppose that $u$ is a differentiable function of the $n$ variables \(x_1,x_2,\cdots,x_n\) and each \(x_j\) is a differentiable function of the $m$ variables \(t_1,t_2,\cdots,t_m\). Then $u$ is a function of \(t_1,t_2,\cdots,t_m\) and \[\pdv{u}{t_i}=\pdv{u}{x_1}\pdv{x_1}{t_i}+\pdv{u}{x_2}\pdv{x_2}{t_i}+\cdots+\pdv{u}{x_n}\pdv{x_n}{t_i}\] for each \(i=1,2,\cdots,m\)
        \1 Implicit diffentiation: \[\dfrac{dy}{dx}=-\dfrac{\pdv{F}{x}}{\pdv{F}{y}}=-\dfrac{F_x}{F_y}\] where \(y=f(x)\) and \(F(x,f(x))=0\)
        \1 \[\pdv{z}{x}=-\dfrac{\pdv{F}{x}}{\pdv{F}{z}}\qquad\pdv{z}{y}=-\dfrac{\pdv{F}{y}}{\pdv{F}{z}}\]


    \end{outline}
    \subsection{Directional Derivatives and the Gradient Vector}
    \begin{outline}
        \1 The directional derivative of $f$ at \((x_0,y_0)\) in the direction of a unit vector \(\vb{u}=\langle a,b\rangle \) is \[D_{\vb{u}}f(x_0,y_0)=\lim_{h\to 0}\dfrac{f(x_0+ha,y_0+hb)-f(x_0,y_0)}{h}\] if this limit exists. 
        \1 Theorem: If $f$ is a differentiable function of $x$ and $y$, then $f$ has a directional derivative in the direction of any unit vector \(\vb{u}=\langle a,b\rangle\) and \[D_{\vb{u}}f(x,y)=f_x(x,y)a+f_y(x,y)b\]
            \2 If the unit vector \(\vb{u}\) makes an angle \(\theta\) with the positive $x$-axis, then we can write \(\vb{u}=\langle\cos\theta,\sin\theta\rangle\) and the previous eqn becomes: \[D_{\vb{u}}f(x,y)=f_x(x,y)\cos\theta+f_y(x,y)\sin\theta\]
        \1 If $f$ is a function of two variables $x$ and $y$, then the gradient of $f$ is the vector function \(\grad{f}\) defined by \[\grad{f(x,y)}=\langle f_x(x,y),f_y(x,y)\rangle=\pdv{f}{x}\vb{i}+\pdv{f}{y}\vb{j}\]
        \1 The equation of the directional derivative of a differentiable function can thus be written as: \[D_{\vb{u}}f(x,y)=\grad{f(x,y)}\cdot\vb{u}\]
        \1 The directional derivative of $f$ at \((x_0,y_0,z_0)\) in the direction of a unit vector \(\vb{u}=\langle a,b,c\rangle\) is \[D_{\vb{u}}f(x_0,y_0,z_0)=\lim_{h\to 0}\dfrac{f(x_0+ha,y_0+hb,z_0+hc)-f(x_0,y_0,z_0)}{h}\] if this limit exists. 
        \1 Using vector notation: \[D_{\vb{u}}f(\vb{x}_0)=\lim_{h\to 0}\dfrac{f(\vb{x_0}+h\vb{u})-f(\vb{x}_0)}{h}\]
        \1 For a function of three variables, the gradient vector: \[\grad{f}=\langle f_x,f_y,f_z\rangle=\pdv{f}{x}\vb{i}+\pdv{f}{y}\vb{j}+\pdv{f}{z}\vb{k}\]
        \1 The directional derivative of a function of three variables: \[D_{\vb{u}}f(x_0,y_0,z_0)=\grad{f(x,y,z)\cdot\vb{u}}\]
        \1 Theorem: Suppose $f$ is a differentiable function of two or three variables. The maximum value of the directional derivative \(D_{\vb{u}}f(\vb{x})\) is \(|\grad f(\vb{x})|\) and it occurs when \(\vb{u}\) has the same direction as the gradient vector \(\grad f(\vb{x})\). 
        \1 The tangent plane to a level surface \(F(x,y,z)=k\) at \(P(x_0,y_0,z_0)\) is the plane that passes through $P$ and has normal vector \(\grad F(x_0,y_0,z_0)\). The equation of the tangent plane is thus: \[F_x(x_0,y_0,z_0)(x-x_0)+F_y(x_0,y_0,z_0)(y-y_0)+F_z(x_0,y_0,z_0)(z-z_0)=0\]
        \1 The normal line to the surface $S$ at $P$ is the line passing through $P$ and perpendicular to the tangent plane. The direction of the normal line is therefore given by the gradient vector \(\grad F(x_0,y_0,z_0)\); its symmetric equations are given by: \[\dfrac{x-x_0}{F_x(x_0,y_0,z_0)}=\dfrac{y-y_0}{F_y(x_0,y_0,z_0)}=\dfrac{z-z_0}{F_z(x_0,y_0,z_0)}\]

    \end{outline}
    \subsection{Maximum and Minimum Values}
    \begin{outline}
        \1 A function of two variables has a local maximum at \((a,b)\) if \(f(x,y)\leq f(a,b)\) when \((x,y)\) is near \((a,b)\). [This means that \(F(x,y)\leq f(a,b)\) for all points \((x,y)\) in some disk with center \((a,b)\).] The number \(f(a,b)\) is called a local maximum value. If \(f(x,y)\geq f(a,b)\) when \((x,y)\) is near \((a,b)\), then $f$ has a local minimum at \((a,b)\) and \(f(a,b)\) is a local minimum value. 
        \1 Theorem: If $f$ has a local maximum or minimum at \((a,b)\) and the first order partial derivatives of $f$ exist there, then \(f_x(a,b)=0\) and \(f_y(a,b)=0\). 
        \1 Second derivatives test: Suppose the second partial derivatives of $f$ are continuous on a disk with center \((a,b)\), and suppose that \(f_x(a,b)=0\) and \(f_y(a,b)=0\) [that is, \((a,b)\) is a critical point of $f$]. Let \[D=D(a,b)=f_{xx}(a,b)f_{yy}(a,b)-[f_{xy}(a,b)]^2\]
            \2 If \(D>0\) and \(f_{xx}(a,b)>0\), then \(f(a,b)\) is a local minimum. 
            \2 If \(D>0\) and \(f_{xx}(a,b)<0\), then \(f(a,b)\) is a local maximum. 
            \2 If \(D<0\), the \(f(a,b)\) is not a local maximum or minimum. 
        \1 Extreme value theorem for functions of two variables: If $f$ is continuous on a closed, bounded set $D$ in \(\mathbb{R}^2\), then $f$ attains an absolute maximum value \(f(x_1,y_1)\) and an absolute minimum value \(f(x_2,y_2)\) at some points \((x_1,y_1)\) and \((x_2,y_2)\) in $D$. 
        \1 To find the absolute maximum and minimum values of a continuous function $f$ on a closed, bounded set $D$:
    \0 
        \begin{enumerate}
            \item Find the values of $f$ at the critical points of $f$ in $D$. 
            \item Find the extreme values of $f$ on the boundary of $D$. 
            \item The largest of the values from steps one and 2 is the absolute maximum value; the smallest of these values is the absolute minimum value. 
        \end{enumerate}

    \end{outline}
    \subsection{Lagrange Multipliers}
    \begin{outline}
        \1 Method of Lagrange Multipliers: To find the maximum and minimum values of \(f(x,y,z)\) subject to the constraint \(g(x,y,z)=k\) [assuming that these extreme values exist and \(\grad{g}\neq \vb{0}\) on the surface \(g(x,y,z)=k\)]: 
        \begin{enumerate}
            \item Find all values of \(x,y,z\), and \(\lambda\) such that \[\grad f(x,y,z)=\lambda\grad g(x,y,z)\] and \[g(x,y,z)=k\]
            \item Evaluate $f$ at all the points \((x,y,z)\) that result from step 1. The largest of these values is the maximum value of $f$; the smallest is the minimum value of $f$. 
        \end{enumerate}
        \1 With two constraints, \(g(x,y,z)=k\) and \(h(x,y,z)=c\), there exist Lagrange Multipliers, constants \(\lambda\) and \(\mu\) such that \[\grad f(x_0,y_0,z_0)=\lambda\grad g(x_0,y_0,z_0)+\mu\grad h(x_0,y_0,z_0)\]
    \end{outline}

\end{document}