\documentclass{article}
\title{Chapter 15 Notes - MC} % title - chapter number
\author{John Yang} 
\usepackage{amsmath}
\usepackage[margin=1in, letterpaper]{geometry}
\usepackage{outlines}
\setcounter{section}{+14} % chapter number minus 1
\usepackage{mathtools}
\DeclarePairedDelimiter\set\{\}
\usepackage{hyperref}
\hypersetup{
	colorlinks=true,
	linkcolor=blue,
	filecolor=magenta,      
	urlcolor=cyan,
}
\usepackage{tocloft}
\renewcommand\cftsecfont{\normalfont}
\renewcommand\cftsecpagefont{\normalfont}
\renewcommand{\cftsecleader}{\cftdotfill{\cftsecdotsep}}
\renewcommand\cftsecdotsep{\cftdot}
\renewcommand\cftsubsecdotsep{\cftdot}

\begin{document}
    \maketitle
    \tableofcontents
    \section{Partial Derivatives} % chapter title
    \subsection{Functions of Several Variables} % section topic
    \begin{outline}
        
    \end{outline}
    \subsection{Limits and Continuity}
    \begin{outline}
        
    \end{outline}
    \subsection{Partial Derivatives}
    \begin{outline}
        
    \end{outline}
    \subsection{Tangent Planes and Linear Approximations}
    \begin{outline}
        
    \end{outline}
    \subsection{The Chain Rule}
    \begin{outline}
        
    \end{outline}
    \subsection{Directional Derivatives and the Gradient Vector}
    \begin{outline}
        
    \end{outline}
    \subsection{Maximum and Minimum Values}
    \begin{outline}
        
    \end{outline}
    \subsection{Lagrange Multipliers}
    \begin{outline}
        
    \end{outline}

\end{document}