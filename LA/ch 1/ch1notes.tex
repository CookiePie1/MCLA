\documentclass{article}
\title{Chapter 1 Notes - LA} % title - chapter number
\author{John Yang}
\usepackage{amsmath}
\usepackage{amssymb}
\usepackage{mathrsfs}
\usepackage{mathtools}
\usepackage[margin=1in, letterpaper]{geometry}
\usepackage{outlines}
\setcounter{section}{+0} % chapter number minus 1
\usepackage{mathtools}
\DeclarePairedDelimiter\set\{\}
\usepackage{hyperref}
\hypersetup{
	colorlinks=true,
	linkcolor=blue,
	filecolor=magenta,      
	urlcolor=cyan,
}
\usepackage{tocloft}
\renewcommand\cftsecfont{\normalfont}
\renewcommand\cftsecpagefont{\normalfont}
\renewcommand{\cftsecleader}{\cftdotfill{\cftsecdotsep}}
\renewcommand\cftsecdotsep{\cftdot}
\renewcommand\cftsubsecdotsep{\cftdot}

\begin{document}
    \maketitle
    \tableofcontents
    \section{Vectors} % chapter title
    \subsection{The Geometry and Algebra of Vectors}
    \begin{outline}
        \1 A vector is a directed line segment that corresponds to a displacement from one point A to another point B. 
        \1 Column vectors and row vectors are different ways to express the same thing: \[[3,2]=\begin{bmatrix}3 \\ 2\end{bmatrix}\]
        \1 The point is that components of vectors are ordered. 
        \1 Two vectors are equal if they have the same magnitude and direction. Two vectors can still be equal if they have different initial and terminal points. 
        \1 Standard position of a vector - when the initial point is at the origin. 
        \1 Sum \(\mathbf u+\mathbf v=[u_1+v_1,u_2+v_2]\)
        \1 Place vectors from head to tail. 
        \1 Scalar multiples: \(c\mathbf v=[cv_1,cv_2]\) aka scaling a vector
        \1 Subtraction is just adding the negative. 
        \1 Properties of vectors in \(\mathbb{R}^n\): let \(\mathbf u\), \(\mathbf v\), and \(\mathbf w\) be vectors in \(\mathbb{R}^n\) and let $c$ and $d$ be scalars. Then: 
            \2 \(\mathbf u+\mathbf v=\mathbf v+\mathbf u\)
            \2 \((\mathbf u+\mathbf v)+\mathbf w=\mathbf u+(\mathbf v+\mathbf w)\)
            \2 \(\mathbf u+\mathbf 0=\mathbf u\)
            \2 \(\mathbf u+(-\mathbf u)=0\)
            \2 \(c(\mathbf u+\mathbf v)=c\mathbf u+c\mathbf v\)
            \2 \((c+d)\mathbf u=c\mathbf u+d\mathbf u\)
            \2 \(c(d\mathbf u)=(cd)\mathbf u\)
            \2 \(1\mathbf u=\mathbf u\)
        \1 A vector $\mathbf v$ is a linear combination of vectors \(\mathbf v_1,\mathbf v_2,\cdots,\mathbf v_k\) if there are scalars \(c_1,c_2,\cdots,c_k\) such that \(\mathbf v=c_1\mathbf v_1+c_2\mathbf v_2+\cdots+c_k\mathbf v_k\). Those scalars are called the coefficients of the linear combination. 
        \1 Binary vectors - the components are either 0 or 1. 
        \1 Modulus function - divide by a given number and you're left with the remainder. 

    \end{outline}
    \subsection{Length and Angle: the Dot Product}
    \begin{outline}
        \1 dot product: If \[\mathbf u=\begin{bmatrix}u_1\\u_2\\\vdots\\u_n\end{bmatrix}\quad\text{and}\quad\mathbf v=\begin{bmatrix}v_1\\v_2\\\vdots\\v_n\end{bmatrix}\] then the dot product of \(\mathbf u\cdot\mathbf v\) of \(\mathbf u\) and \(\mathbf v\) is defined by \[\mathbf u\cdot\mathbf v=u_1v_1+u_2v_2+\cdots+u_nv_n\]
        \1 properties of dot product: let \(\mathbf u\), \(\mathbf v\), and \(\mathbf w\) be vectors in \(\mathbb R^n\) and let $c$ be a scalar. Then: 
            \2 \(\mathbf{u\cdot v}=\mathbf{v\cdot u}\)
            \2 \(\mathbf{u\cdot}(\mathbf{v+w})=\mathbf{u\cdot v+u\cdot w}\)
            \2 \((c\mathbf u)\mathbf{\cdot v}=c(\mathbf{u\cdot v})\)
            \2 \(\mathbf{u\cdot u\geq 0}\) and \(\mathbf{u\cdot u}=0\) IFF \(\mathbf u=\mathbf 0\)
            \2 Length or norm of a vector \(\mathbf v=\begin{bmatrix}v_1\\v_2\\\vdots\\v_n\end{bmatrix}\) in \(\mathbb R^n\) is the nonnegative scalar \(||\mathbf v||\) defined by \[||\mathbf v||=\sqrt{\mathbf{v\cdot v}}=\sqrt{v_1^2+v_2^2+\cdots+v_n^2}\]
        \1 Normalizing a vector means finding the unit vector. 
        \1 Cauchy-Schwarz Inequality: For all vectors \(\mathbf u\) and \(\mathbf v\) in \(\mathbb R^n\), \[|\mathbf{u\cdot v}|\leq ||\mathbf u||||\mathbf v||\]
        \1 Triangle inequality: for all vectors \(\mathbf u\) and \(\mathbf v\) and \(\mathbb R^n\), \[||\mathbf u+\mathbf v||\leq||\mathbf u||+||\mathbf v||\]
        \1 Distance between two vectors is defined by \[d(\mathbf u,\mathbf v)=||\mathbf u-\mathbf v||\]
        \1 Two vectors are orthogonal if \(\mathbf u\cdot\mathbf v=0\)
        \1 For all vectors \(\mathbf u\) and \(\mathbf v\) in \(\mathbb R^n\), \(||\mathbf u+\mathbf v||^2=||\mathbf u||^2+||\mathbf v||^2\) IFF \(\mathbf u\) and \(\mathbf v\) are orthogonal. 
        \1 If \(\mathbf u\) and \(\mathbf v\) are vectors in \(\mathbb R^n\) and \(\mathbf u\neq\mathbf 0\), then the projection of \(\mathbf v\) onto \(\mathbf u\) is the vector defined by \[\text{proj}_{\mathbf u}(\mathbf v)=\left(\dfrac{\mathbf{u\cdot v}}{\mathbf{u\cdot u}}\right)\mathbf u\]

    \end{outline}
    \subsection{Lines and Planes}
    \begin{outline}
        \1 Normal form of the equation of a 2D line: \[\mathbf{n\cdot(x-p)=0}\qquad\text{or}\qquad\mathbf{n\cdot x=n\cdot p}\] where \(\mathbf p\) is a specific point on the line and \(\mathbf{n\neq 0}\) is a normal vector for the line. 
        \1 The general form of the equation of the line is \(ax+by=c\) where \(\mathbf n=\begin{bmatrix}a\\b\end{bmatrix}\) is a normal vector for the line. 
        \1 The vector form of the equation of a 2D or 3D line is \[\mathbf{x=p}+t\mathbf{d}\] where \(\mathbf p\) is a specific point on the line and \(\mathbf{d\neq 0}\) is a direction vector for the line. The equations corresponding to the components of the vector form of the equations are called parametric equations of the line. 
        \1 Normal form of the equation of a plane \(\mathscr{P}\) in \(\mathbb R^3\) is \[\mathbf{n\cdot(x-p)}=0\qquad\text{or}\qquad\mathbf{n\cdot x=n\cdot p}\] where \(\mathbf p\) is a specific point on \(\mathscr P\) and \(\mathbf{n\neq 0}\) is a normal vector for \(\mathscr P\). 
        \1 The general form of the equation of \(\mathscr P\) is \(ax+by+cz=d\), where \(\mathbf n=\begin{bmatrix}a\\b\\c\end{bmatrix}\) is a normal vector for \(\mathscr P\). 
        \1 The vector form of the equation of a plane \(\mathscr P\) in \(\mathbb R^3\) is \[\mathbf{x=p}+s\mathbf u+t\mathbf v\] where \(\mathbf p\) is a point on \(\mathscr P\) and \(\mathbf u\) and \(\mathbf v\) are direction vectors for \(\mathscr P\) (\(\mathbf u\) and \(\mathbf v\) are nonzero and parallel to \(\mathscr P\), but not parallel to each other). The equations corresponding to the components of the vector form of the equation are called parametric equations of \(\mathscr P\). 
        \1 Summary of equations of 2D lines:
            \2 Normal form: \(\mathbf{n\cdot x=n\cdot p}\)
            \2 General form: \(ax+by=c\)
            \2 Vector form: \(\mathbf{x=p}+t\mathbf{d}\)
            \2 Parametric form: \[\begin{cases}x=p_1+td_1\\y=p_2+td_2\end{cases}\]
        \1 Summary of equations of 3D lines:
            \2 Normal form: \[\begin{cases}
                \mathbf{n_1\cdot x=n_1\cdot p_1} \\
                \mathbf{n_2\cdot x=n_2\cdot p_2}
            \end{cases}\]
            \2 General form: \[\begin{cases}
                a_1x+b_1y+c_1z=d_1\\
                a_2x+b_2y+c_2z=d_2
            \end{cases}\]
            \2 Vector form: \(\mathbf{x=p}+t\mathbf d\)
            \2 Parametric form: \[\begin{cases}
                x=p_1+td_1\\
                y=p_2+td_2\\
                z=p_3+td_3
            \end{cases}\]
        \1 Summary of equations of 3D planes: 
            \2 Normal form: \(\mathbf{n\cdot x=n\cdot p}\)
            \2 General form: \(ax+by+cz=d\)
            \2 Vector form: \(\mathbf{x=p}+s\mathbf u+t\mathbf v\)
            \2 Parametric form: \[\begin{cases}
                x=p_1+su_1+tv_1\\
                y=p_2+su_2+tv_2\\
                z=p_3+su_3+tv_3
            \end{cases}\]
    \end{outline}
    \subsection{Applications}
    \begin{outline}
        \1 Force vectors: if the resultant net force is zero, the system is in equilibrium. 
        \1 Resolve into components to work with the vectors. 
    \end{outline}
\end{document}