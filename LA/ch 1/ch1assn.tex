\documentclass{article}
\title{Chapter 1 Assignment - LA} % title - chapter number
\author{John Yang}
\usepackage{amsmath}
\usepackage{amssymb}
\usepackage{mathrsfs}
\usepackage[margin=1in, letterpaper]{geometry}
\usepackage{outlines}
\setcounter{section}{+0} % chapter number minus 1
\usepackage{mathtools}
\DeclarePairedDelimiter\set\{\}
\usepackage{hyperref}
\usepackage{physics}
\hypersetup{
	colorlinks=true,
	linkcolor=blue,
	filecolor=magenta,      
	urlcolor=cyan,
}
\usepackage{tocloft}
\renewcommand\cftsecfont{\normalfont}
\renewcommand\cftsecpagefont{\normalfont}
\renewcommand{\cftsecleader}{\cftdotfill{\cftsecdotsep}}
\renewcommand\cftsecdotsep{\cftdot}
\renewcommand\cftsubsecdotsep{\cftdot}

\begin{document}
    \maketitle
    \tableofcontents
    \section{Vectors} % chapter title
    \subsection{Questions}
    \begin{outline}[enumerate]
        \1 What is a vector? What are its properties?
            \2 A directed line segment that corresponds to a displacement from one point A to another point B
            \2 That is, a quantity with both magnitude and direction 
        \1 How many components can a vector have? In how many distinct ways can we write a vector in LA?
            \2 It can have infinitely many components. Can be expressed in row form or column form. 
        \1 When we change the coordinate frame, what about a vector changes, what about a vector remains the same (invariance)?
            \2 The magnitude stays the same, the true direction stays the same, but the components change. 
        \1 What are the algebraic properties of vectors? What is vector algebra? Give analytic and geometric methods of applying vector algebra operations to vectors. 
            \2 \(\mathbf u+\mathbf v=\mathbf v+\mathbf u\)
            \2 \((\mathbf u+\mathbf v)+\mathbf w=\mathbf u+(\mathbf v+\mathbf w)\)
            \2 \(\mathbf u+\mathbf 0=\mathbf u\)
            \2 \(\mathbf u+(-\mathbf u)=0\)
            \2 \(c(\mathbf u+\mathbf v)=c\mathbf u+c\mathbf v\)
            \2 \((c+d)\mathbf u=c\mathbf u+d\mathbf u\)
            \2 \(c(d\mathbf u)=(cd)\mathbf u\)
            \2 \(1\mathbf u=\mathbf u\)
            \2 Vector algebra - manipulating vectors to find new ones or solve equations. Take addition, subtraction and multiplication of components analytically, or draw them out geometrically. 

        \1 What is a linear combination? What is a coordinate, a coordinate system?
            \2 A vector $\mathbf v$ is a linear combination of vectors \(\mathbf v_1,\mathbf v_2,\cdots,\mathbf v_k\) if there are scalars \(c_1,c_2,\cdots,c_k\) such that \(\mathbf v=c_1\mathbf v_1+c_2\mathbf v_2+\cdots+c_k\mathbf v_k\). Those scalars are called the coefficients of the linear combination. 
        \1 What is the dot product of vectors? What are its properties?
            \2 dot product: If \[\mathbf u=\begin{bmatrix}u_1\\u_2\\\vdots\\u_n\end{bmatrix}\quad\text{and}\quad\mathbf v=\begin{bmatrix}v_1\\v_2\\\vdots\\v_n\end{bmatrix}\] then the dot product of \(\mathbf u\cdot\mathbf v\) of \(\mathbf u\) and \(\mathbf v\) is defined by \[\mathbf u\cdot\mathbf v=u_1v_1+u_2v_2+\cdots+u_nv_n\]
            \2 \(\mathbf{u\cdot v}=\mathbf{v\cdot u}\)
            \2 \(\mathbf{u\cdot}(\mathbf{v+w})=\mathbf{u\cdot v+u\cdot w}\)
            \2 \((c\mathbf u)\mathbf{\cdot v}=c(\mathbf{u\cdot v})\)
            \2 \(\mathbf{u\cdot u\geq 0}\) and \(\mathbf{u\cdot u}=0\) IFF \(\mathbf u=\mathbf 0\)

        \1 What is the length (norm) of a vector? What are its properties? Does the norm of a vector depend on the coordinate system in use? That is, is it or is it not an invariant under coordinate transformations?
            \2 Length or norm of a vector \(\mathbf v=\begin{bmatrix}v_1\\v_2\\\vdots\\v_n\end{bmatrix}\) in \(\mathbb R^n\) is the nonnegative scalar \(||\mathbf v||\) defined by \[||\mathbf v||=\sqrt{\mathbf{v\cdot v}}=\sqrt{v_1^2+v_2^2+\cdots+v_n^2}\]

        \1 What is the Cauchy-Schwartz inequality?
            \2 For all vectors \(\mathbf u\) and \(\mathbf v\) in \(\mathbb R^n\), \[|\mathbf{u\cdot v}|\leq ||\mathbf u||||\mathbf v||\]
        \1 What is the Triangle inequality?
            \2 for all vectors \(\mathbf u\) and \(\mathbf v\) and \(\mathbb R^n\), \[||\mathbf u+\mathbf v||\leq||\mathbf u||+||\mathbf v||\]
        \1 What is the vector orthogonality?
            \2 Two vectors are orthogonal if \(\mathbf u\cdot\mathbf v=0\)
        \1 What is Pythagoras' Theorem in LA?
            \2 For all vectors \(\vb u\) and \(\vb v\) in \(\mathbb R^n\), \(||\vb u+\vb v||^2=||\vb u||^2+||\vb v||^2\) IFF \(\vb u\) and \(\vb v\) are orthogonal. 
        \1 What is a line? What vectors describe a given line? What are the unit tangent vector \(\vec T\), the unit normal vector \(\vec N\), the unit binormal vector \(\vec B\) to a line, and the curvature \(\kappa\) of the line?
            \2 A line is a set of points in space. See paper worksheet. 
        \1 Given a line \(\ell\) what are its normal form, general form, vector form, and parametric equations?
            \2 normal: \(\vb n\cdot (\vb x-\vb p)=0\)
            \2 general: \(ax+by=c\)
            \2 vector: \(\vb x=\vb p+t\vb d\)
            \2 Parametric equations: \(x=f(t),y=f(t)\)
        \1 What is a plane? What vector describes a given plane?
            \2 Normal vector describes the plane. The plane is the set of all points that satisfy the general equation of the plane (?). 
        \1 Given a plane \(\mathcal{P}\) what are its normal form, general form, vector form, and parametric equations?
            \2 \(\vb n\cdot\vb x=\vb n\cdot\vb p\), \(ax+by+cz=d\), \(\vb x=\vb p+s\vb u+t\vb v\), \(\begin{cases}
                x=p_1+su_1+tv_1\\
                y=p_2+su_2+tv_2\\
                z=p_3+su_3+tv_3
            \end{cases}\)
        \1 Derive an equation for the shortest distance \(d(B,\ell)\) from a point \((x_0,y_0)\) to a line \(ax+by=c\)
            \2 
        \1 Derive an equaiton for the shortest distance \(d(B,\mathcal{P})\) from a point \((x_0,y_0,z_0)\) to a plane \(ax+by+cz=d\)
            \2 
        \1 What is the binary code?
            \2 Only two values, 0 and 1. Consists of a set of binary code vectors that contain some message. 
        \1 What is an error detecting code? How does it work?
            \2 
        \1 What is modular arithmetic?
            \2 When you divide by a given number and only keep the remainder
    \end{outline}
    % Book exercises 1.1: 3, 11, 12, 13, 16, 19, 21, 24
    % 1.2 5, 11, 15, 19, 25, 32, 35, 37, 41, 45, 47-68
    % 1.3 9, 11, 13, 16, 17, 25, 39, 40-43, 47, 48
    % 1.4 EOO
    % Ch review questions as ch assessment
\end{document}