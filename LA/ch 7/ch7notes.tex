\documentclass{article}
\title{Chapter 7 Notes - LA} % title - chapter number
\author{John Yang}
\usepackage{amsmath}
\usepackage[margin=1in, letterpaper]{geometry}
\usepackage{outlines}
\setcounter{section}{+6} % chapter number minus 1
\usepackage{mathtools}
\usepackage{amssymb}
\usepackage{physics}
\usepackage{stackrel}
\DeclarePairedDelimiter\set\{\}
\usepackage{hyperref}
\usepackage{mathrsfs}
\hypersetup{
	colorlinks=true,
	linkcolor=blue,
	filecolor=magenta,      
	urlcolor=cyan,
}
\usepackage{tocloft}
\renewcommand\cftsecfont{\normalfont}
\renewcommand\cftsecpagefont{\normalfont}
\renewcommand{\cftsecleader}{\cftdotfill{\cftsecdotsep}}
\renewcommand\cftsecdotsep{\cftdot}
\renewcommand\cftsubsecdotsep{\cftdot}

\begin{document}
    \maketitle
    \tableofcontents
    \section{Distance and Approximation} % chapter title
    \subsection{Inner Product Spaces} 
    \begin{outline}
        \1 An inner product on a vector space $V$ is an operation that assigns to every pair of vectors \(\vb u\) and \(\vb v\) in $V$ a real number \(\langle\vb u,\vb v\rangle\) such that the following properties hold for all vectors \(\vb u\), \(\vb v\), and \(\vb w\) in $V$ and all scalars $c$:
            \2 \(\langle\vb u,\vb v\rangle=\langle\vb v,\vb u\rangle\)
            \2 \(\langle\vb u,\vb v+\vb w\rangle=\langle\vb u,\vb v\rangle+\langle\vb u,\vb w\rangle\)
            \2 \(\langle c\vb u,\vb v\rangle=c\langle\vb u,\vb v\rangle\)
            \2 \(\langle\vb u,\vb u\rangle\geq 0\) and \(\langle\vb u,\vb u\rangle=0\) IFF \(\vb u=\vb 0\)
        \1 A vector space with an inner product is called an inner product space. 
        \1 Let \(\vb u\), \(\vb v\), and \(\vb w\) be vectors in an inner product space $V$ and let \(c\) be a scalar. 
            \2 \(\langle\vb u+\vb v,\vb w\rangle=\langle\vb u,\vb w\rangle+\langle\vb v+\vb w\rangle\)
            \2 \(\langle\vb u,c\vb v\rangle=c\langle\vb u,\vb v\rangle\)
            \2 \(\langle\vb u,\vb 0\rangle=\langle\vb 0,\vb v\rangle=0\)
        \1 Let \(\vb u\) and \(\vb v\) be vectors in an inner product space $V$. 
            \2 The length (or norm) of \(\vb v\) is \(||\vb v||=\sqrt{\langle\vb v,\vb v\rangle}\). 
            \2 The distance between \(\vb u\) and \(\vb v\) is \(d(\vb u,\vb v)=||\vb u-\vb v||\)
            \2 \(\vb u\) and \(\vb v\) are orthogonal if \(\langle\vb u,\vb v\rangle=0\). 
        \1 Pythagoras' Theorem: Let \(\vb u\) and \(\vb v\) be vectors in an inner product space $V$. Then \(\vb u\) and \(\vb v\) are orthogonal IFF \[||\vb u+\vb v||^2=||\vb u||^2+||\vb v||^2\]
        \1 The Cauchy-Schwarz Inequality: Let \(\vb u\) and \(\vb v\) be vectors in an inner product space $V$. Then \[|\langle\vb u,\vb v\rangle|\leq||\vb u||||\vb v||\] with equality holding IFF $\vb u$ and $\vb v$ are scalar multiples of each other. 
	\1 The triangle inequality: Let $\vb u$ and $\vb v$ be vectors in an inner product space $V$. Then \[||\vb u+\vb v||\leq||\vb u||+||\vb v||\]
    \end{outline}
    \subsection{Norms and Distance Functions} 
    \begin{outline}
        \1 A norm on a vector space $V$ is a mapping that associates with each vector $\vb v$ a real number $||\vb v||$, called the norm of $\vb v$, such that the following properties are satisfied for all vectors $\vb u$ and $\vb v$ and all scalars $c$:
		\2 \(||\vb v||\geq 0\), and \(||\vb v||=0\) IFF \(\vb v=\vb 0\)
		\2 \(||c\vb v||=|c|||\vb v||\)
		\2 \(||\vb u+\vb v||\leq||\vb u||+||\vb v||\)
	\1 A vector space with a norm is called a normed vector space. 
	\1 We define a distance function for any norm as: \[d(\vb u,\vb v)=||\vb u-\vb v||\]
	\1 Let $d$ be a distance function defined on a normed linear space $V$. The following properties hold for all vectors $\vb u$, $\vb v$, and $\vb w$ in $V$:
		\2 \(d(\vb u,\vb v)\geq 0\), and \(d(\vb u,\vb v)=0\) IFF \(\vb u=\vb v\)
		\2 
    \end{outline}
    \subsection{Least Squares Approximation} 
    \begin{outline}
        \1 
    \end{outline}
    \subsection{The Singular Value Decomposition} 
    \begin{outline}
        \1 
    \end{outline}
    \subsection{Applications} 
    \begin{outline}
        \1 
    \end{outline}
    
\end{document}
